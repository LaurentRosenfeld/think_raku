\chapter{Some Final Advice}

\begin{quote}
\raggedleft 
\emph{Everyone knows that debugging is twice as hard as writing \\ 
a program in the first place. So if you're as clever as you \\
can be when you write it, how will you ever debug it? }\\
--- Brian Kernighan, "The Elements of Programming Style".
\end{quote}
\index{Kernighan, Brian}

\section{Make it Clear, Keep it Simple}

Writing a real-life program is not the same thing as learning 
the art of programming or learning a new language.

Because the goal of this book is to lead you to learn 
more advanced concepts or new syntax, I have often been 
pushing new ways of doing things. But this does not mean that 
you should try to pack your most advanced knowledge into 
each of your programs. Quite the contrary.

The rule of thumb is ``KISS'': keep it simple, stupid. The 
KISS engineering principle (originated in the US Navy 
around 1960) states that most systems work best if they 
are kept simple rather than made complicated; therefore 
simplicity should be a key goal in design and unnecessary 
complexity should be avoided. This does not mean, however, 
that you should write simplistic code.
\index{KISS: keep it simple, stupid}

As an example, if you're looking for a literal substring 
within a string, use the simple {\tt index} built-in function, 
rather than firing the regex engine for that. Similarly, 
if you know the position and length of the substring, then 
use the {\tt substr} function. But if you need a more ``fuzzy'' 
match with perhaps some alternatives or a character class, then 
a regex is very likely the right tool.
\index{index}
\index{substr}
\index{regex}

Another related tenet is ``YAGNI'': you aren't gonna need 
it. This acronym comes from a school of programming 
known as ``extreme programming'' (XP). Even if you don't 
adhere to all the XP principles, this idea is quite sound 
and well-founded: don't add any functionality until it is 
really needed.
\index{YAGNI: you aren't gonna need it}

Try to make your programs as clear as possible, and 
as simple as you can. Use more advanced concepts if 
you have to, but don't do it for the sake of showing how 
masterful you are. Don't try to be clever or, at least, 
don't be \emph{too} clever.

Remember that code is not only used by the compiler, but 
is also by humans. Think about them.

Think about the person who will have maintain your code. 
As some people like to put it: ``Always code as if the person 
who ends up maintaining your code is a violent psychopath 
who knows where you live.'' And, if nothing else will convince 
you, remember that the person maintaining your code might be 
you a year from now. You may not remember then 
how that neat trick you used really works.

A final quote from Edsger Dijkstra on this subject: 
``Simplicity is prerequisite for reliability.''
\index{Dijkstra, Edsger}
\index{simplicity}

\section{Dos and Don'ts}

\begin{description}

\item[Don't repeat yourself (DRY):] Avoid code duplication. 
If you have the same code in different places of your 
program, then something is most likely wrong. Maybe 
the repeated code should go into a loop or a separate 
subroutine, or perhaps even in a module or a library. 
Remember that copy and paste is a source of evil.
\index{DRY: don't repeat yourself}

\item[Don't reinvent the wheel:] Use existing libraries 
and modules when you can; it is likely that they have 
been thoroughly tested and will work better than the 
quick fix you're about to write. The Perl~6 ecosystem 
has a large and growing collection of software modules 
(see \url{modules.perl6.org}) that you can use in your 
programs.
\index{reinventing the wheel}

\item[Use meaningful identifiers] If your variables, 
subroutines, methods, classes, grammars, modules, 
and programs have sensible 
names that convey clearly what they are or what they do, 
then your code will be clearer and might need fewer 
comments. Very short names like \verb'$i' or \verb'$n' are 
usually fine for loop variables, but pretty much anything else needs 
a name that clearly explains what the content or the purpose is. 
Names like \verb'$array' or \verb'%hash' may have sometimes been 
used in some examples of this book to indicate more clearly 
the nature of the data structure, but they are not recommended in 
real-life programs. If your hash contains a collection of 
words, call it \verb'%words' or \verb'%word-list', not 
\verb'%hash'. The \% sigil indicates that it is a hash anyway.
\index{meaningful identifier}

\item[Make useful comments and avoid useless ones] A 
comment like this:
\begin{verbatim}
my $count = 1;     # assigns 1 to $count
\end{verbatim}
is completely useless. In general, your comments should 
explain neither what your code is doing nor even how it is doing 
it (this should be obvious if your code is clear), but 
rather \emph{why} you are doing that: perhaps you should refer 
to a math theorem, a law of physics, a design decision, or 
a business rule.
\index{comment}

\item[Remove dead code and code scaffolding] Even when 
writing new code, you may at some point create variables 
that you don't use in the final version of your code. If so, 
remove them; don't let them distract the attention of your 
reader. If you modify an existing program, clean up the place 
after you've changed it. Remember the boy scout's rule: 
leave the place better and cleaner than you found it.
\index{dead code}
\index{scaffolding}

\item[Test aggressively] Nobody can write any piece of 
significant software without having a number of initial 
bugs. Edsger Dijkstra is quoted as saying: ``If 
debugging is the process of removing software bugs,
then programming must be the process of putting them in.'' 
It is unfortunately very true. Even though Dijkstra 
also said that ``Testing shows the presence, not the absence 
of bugs,'' testing is an essential part of software 
development. Write extensive test plans, use them often, and 
update them as the functionality evolves. See Section~\ref{test_module} 
(p.~\pageref{test_module}) for some automated testing tools.
\index{test}
\index{Dijkstra, Edsger}

\item[Avoid premature optimization] In the words of Donald 
Knuth: ``Premature optimization is the source of all evil (or 
at least most of it) in programming.''
\index{premature optimization}
\index{Knuth, Donald}

\item[Don't use magical numbers:] Consider this:
\begin{verbatim}
my $time-left = 31536000;
\end{verbatim}

What is this 31,536,000 number coming out of nowhere? 
There's no way to know just by looking at this line 
of code. Compare with this:

\begin{verbatim}
my $secondsInAYear = 365 * 24 * 60 * 60;
# ...
my $time-left = $secondsInAYear;
\end{verbatim}

Isn't the second version clearer?\footnote{This calculation refers to a \emph{common} year (i.e. a non-leap calendar or civil year). It is not 
the same as various definitions of astronomical years, which are slightly less 
than a quarter of a day longer.} Well, to tell the 
truth, it would be even better to use a constant in 
such a case:
\begin{verbatim}
constant SECONDS-PER-YEAR = 365 * 24 * 60 * 60;
\end{verbatim}
\index{magical number}
\index{constant}

\item[Avoid hardcoded values:] Hard-coded values are bad.
If you have to use some, define them as variables or constants 
at the beginning of your program, and use those variables or 
constants instead. Hard-coded file paths are especially bad. If you 
have to use some, use some variables with relative paths:
\begin{verbatim}
my $base-dir = '/path/to/application/data';
my $input-dir = "$base-dir/INPUT";
my $result-dir = "$base-dir/RESULT";
my $temp-dir = "$base-dir/TEMP";
my $log-dir = "$base-dir/LOG";
\end{verbatim}
At least, if the path must change, you have to change only 
the top code line.
\index{hard-coded value}

\item[Don't ignore errors returned by subroutines or built-in 
functions:] Not all return values are useful; for example, we 
usually don't check the return value of a print statement, but 
that's usually fine because we are interested in the side effect, 
the fact of printing something out to the screen or to a file, 
rather than in the return value. In most other cases, you need 
to know if something went wrong in order to take steps to either 
recover from the error condition, if possible, or to abort the 
program gracefully (e.g., with an informative error message) if 
the error is too serious for the program to continue.
\index{error!ignoring}

\item[Format your code clearly and consistently] The compiler 
might not care about code indentation, but human readers do. 
Your code formatting should help clarify the structure and 
control flow of your programs.
\index{indentation}

\item[Be nice and have fun.]
\index{fun}

\end{description}

\section{Use Idioms}
\index{idiom}

Any language has its own ``best practice'' methods of use. 
These are the idioms that experienced programmers use, ways of doing 
things that have become preferred over time. These idioms are important. 
They protect you from reinventing the wheel. They are also what 
experienced users expect to read; they are familiar 
and enable you to focus on the overall code design rather than
get bogged down in detailed code concerns. They often formalize 
patterns that avoid common mistakes or bugs.

Even though Perl~6 is a relatively new language, a number of such 
idioms have become honed over time. Here are a few of these 
idiomatic constructs~\footnote{When two solutions are suggested, 
the second one is usually the more idiomatic one.}.
\index{idiomatic Perl~6}


\begin{description}
\item[Creating a hash from a list of keys and a list of values] Using slices:
\index{hash slice}

\begin{verbatim}
my %hash; %hash{@keys} = @values;
\end{verbatim}

Using the \emph{zip} operator and a metaoperator with the pair constructor:
\index{zip operator}
\index{metaoperator}
\index{operator!zip}

\begin{verbatim}
my %hash = @keys Z=> @values;
\end{verbatim}

For existence tests, the hash values only need to be true. Here is 
a good way to create a hash from a list of keys:
\begin{verbatim}
my %exists = @keys X=> True;
\end{verbatim}

Or, better yet, use a set:
\begin{verbatim}
my $exists = @keys.Set;
say "exists" if $exists{$object};
\end{verbatim}

\item[Making mandatory attributes (or subroutine parameters):] 
This is a nice way of making a mandatory attribute in a class:
\index{mandatory attribute}
\begin{verbatim}
has $.attr = die "The 'attr' attribute is mandatory";
\end{verbatim}
This code uses the default value mechanism: if a value is supplied, 
then the code for the default value does not run. If no value is 
supplied, then the code dies with the appropriate error message. 
The same mechanism can be used for subroutine parameters.

Or you could use the \verb'is required' trait:

\begin{verbatim}
> class A { has $.a is required }; 
> A.new;
The attribute '$!a' is required, 
but you did not provide a value for it.
\end{verbatim}

You can even pass a explanatory message:

\begin{verbatim}
> class A { has $.a is required("We need it") }; 
> A.new;
The attribute '$!a' is required because We need it,
but you did not provide a value for it.
\end{verbatim}

\item[Iterating over the subscripts of an array] The first 
solution that comes to mind might be:
\index{end method}

\begin{verbatim}
for 0 .. @array.end -> $i {...}
\end{verbatim}

That's fine, but this is probably even better:
\index{keys function or method}

\begin{verbatim}
for @array.keys -> $i {...}
\end{verbatim}

\item[Iterating over the subscripts and values of an array] 
The \verb'.kv' method, in combination with a pointy block 
taking two parameters, allows you to easily iterate over an 
array:
\index{kv function or method}

\begin{verbatim}
for @array.kv -> $i, $value {...}
\end{verbatim}


\item[Printing the number of items in an array] Two possible 
solutions:
\index{elems function or method}

\begin{verbatim}
say +@array; 
# or:
say @array.elems; 
\end{verbatim}

\item[Do something every third time] Use the \verb'%%' 
divisibility operator on the loop variable:
\index{divisibility}
\begin{verbatim}
if $i %% 3 {...}  
\end{verbatim}

\item[Do something \emph{n} times:] Use the right-open range operator:
\index{range operator}

\begin{verbatim}
for 0 ..^ $n {...}
# or, simpler:
for ^$n {...} 
\end{verbatim}

\item[Split a string into words (splitting on space):] 
A method call without an explicit invocant always uses 
the \verb'$_' topical variable as an implicit invocant. Thus, 
assuming the string has been topicalized into \verb'$_':
\index{split function or method}
\index{words function or method}
\index{topical variable}
\index{invocant}

\begin{verbatim}
@words = .split(/\s+/);
# or, simpler:
@words = .words;
\end{verbatim}

\item[An infinite loop] A {\tt loop} statement with no parentheses 
and no argument loops forever: 
\index{infinite loop}
\index{loop!infinite}

\begin{verbatim}
while True {...}
# or, more idiomatic:
loop {...}   
\end{verbatim}

Of course, the body of the {\tt loop} statement must have some 
kind of flow control statement to exit the loop at some point.

\item[Returning the unique elements of a list] The {\tt unique} 
method removes duplicates from the input list:
\index{unique function}

\begin{verbatim}
return @array.unique;
\end{verbatim}

Or, if you know that your list is sorted, you can use the 
{\tt squish} function (which removes adjacent duplicates).
\index{squish function}

\item[Adding up the items of a list] Use the {\tt reduce} 
function or the reduction metaoperator:
\index{reduce function}
\index{reduction operator}
\index{sum function or method}

\begin{verbatim}
my $sum = @a.reduce(* + *);
# or, simpler:
my $sum = [+] @a;
# or yet simpler, using the sum built-in:
my $sum = @a.sum;  
\end{verbatim}

\item[Swapping two variables] Use the \verb'.=' mutating 
method call with the \verb'reverse' function:
\begin{verbatim}
( $x, $y ) =   $y, $x;
# or:
( $x, $y ) .= reverse; # equivalent to: ($x, $y) = ($x, $y).reverse
\end{verbatim}
\index{swapping variables}

\item[Generating random integers between 2 and 6] Use the \verb'..' 
range operator and the {\tt pick} method:
\index{pick function or method}
\index{random number}
\begin{verbatim}
$z = 2 + Int(5.rand);
# or, better:
$z = (2..6).pick;
\end{verbatim}

\item[Count by steps of 3 in an infinite loop] Use the \verb'...' 
sequence operator with the ``*'' whatever star operator and 
a pointy block:
\index{whatever operator}
\index{sequence operator}

\begin{verbatim}
for 3, * + 3 ... * -> $n {...}
# or:
for 3, 6, 9 ... * -> $n {...}    
\end{verbatim}

\item[Loop on a range of values, discounting the range limits:] Use 
the open range operator:
\index{range operator}

\begin{verbatim}
for ($start+1) .. ($end-1) -> $i {...}
# or, better:
for $start ^..^ $end -> $i {...}
\end{verbatim}
\end{description}

\section{What's Next?}

A book like this one can't tell you everything about programming,  
nor about Perl~6. At this point, you should know how to write a 
program to solve an average-difficulty problem, but a lot of work 
has been done in the last decades to solve harder problems. So 
where should you go from here?

Read books about algorithmics, the science of algorithms. Many good 
books exist on the subject, but I especially recommend the following 
two (you should be aware, though, that they are not easy):
\begin{itemize}
\item Thomas H. Cormen, Charles E. Leiserson, Ronald L. Rivest, 
and Clifford Stein, \emph{Introduction to 
Algorithms}, The MIT Press
\item Donald Knuth, \emph{The Art of Computer Programming}, Addison Wesley 
(many volumes, many editions).
\end{itemize}

Read other books about programming, even if they 
target other programming languages or no specific programming 
language. They're likely to have a 
different approach on various points; this will offer you a 
different perspective and perhaps better comprehension, and 
will complement what you have read here. 
Read tutorials, articles, blogs, and forums about programming. 
Participate when you can. Read the \emph{Introduction to 
Perl~6} which exists in eight different languages as of 
this writing (\url{http://perl6intro.com/}). Read the 
official Perl~6 documentation (\url{https://docs.perl6.org}).
\index{Perl~6 documentation}

This book has more than a thousand code examples, which is quite 
a lot, but may not be sufficient if you really want to learn more. 
You should also read code samples written by others. Look for 
open source libraries or modules and try to understand what they 
do and how they do it. Try to use them.

Having said that, I should stress that you can read as many books 
as you want about the theory of swimming, but you'll never know 
swimming until you really get around to doing it. The same is true 
about learning to program and learning a programming 
language. Write new code. Modify existing examples, and see what 
happens. Try new things. Go ahead, be bold, dive into the pool and
swim. The bottom line is: you will really learn by doing.

Learning the art of programming is great fun. Enjoy it!
\index{fun}

