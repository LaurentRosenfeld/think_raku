\chapter{The Way of the Program}

The goal of this book is to teach you to think like a computer
scientist.  This way of thinking combines some of the best features of
mathematics, engineering, and natural science.  Like mathematicians,
computer scientists use formal languages to denote ideas (specifically
computations).  Like engineers, they design things, assembling
components into systems and evaluating tradeoffs among alternatives.
Like scientists, they observe the behavior of complex systems, form
hypotheses, and test predictions.  
\index{problem solving}
\index{formal language}

The single most important skill for a computer scientist is {\bf
problem solving}.  Problem solving means the ability to formulate
problems, think creatively about solutions, and express a solution
clearly and accurately.  As it turns out, the process of learning to
program is an excellent opportunity to practice problem-solving
skills.  That's why this chapter is called, ``The Way of the
Program.''

On one level, you will be learning to program, a useful skill by
itself.  On another level, you will use programming as a means to an
end.  As we go along, that end will become clearer.
\index{programming}


\section{What is a Program?}

A {\bf program} is a sequence of instructions that specifies how to
perform a computation.  The computation might be something
mathematical, such as solving a system of equations or finding the
roots of a polynomial, but it can also be a symbolic computation, such
as searching and replacing text in a document, or something
graphical, like processing an image or playing a video.
\index{program}

The details look different in different languages, but a few basic
instructions appear in just about every language:

\begin{description}

\item[Input] Get data from the keyboard, a file, the network, 
a sensor, a GPS chip or some other device.
\index{input}

\item[Output] Display data on the screen, save it in a
file, send it over the network, act on a mechanical device,  etc.
\index{output}

\item[Math] Perform basic mathematical operations like addition and
multiplication.

\item[Conditional execution] Check for certain conditions and
run the appropriate code.
\index{conditional!execution}

\item[Repetition] Perform some action repeatedly, usually with
some variation.
\index{repetition}

\end{description}

Believe it or not, that's pretty much all there is to it.  Every
program you've ever used, no matter how complicated, is made up of
instructions that look pretty much like these.  So you can think of
programming as the process of breaking a large, complex task
into smaller and smaller subtasks until the subtasks are
simple enough to be performed with one of these basic instructions.
\index{instruction}

\index{abstraction}
\index{subtask}
Using or calling these subtasks makes it possible to create 
various levels of \emph{abstraction}. You have probably 
been told that computers only use 0's and 1's at the 
lowest level; but we usually don't have to worry about that. 
When we use a word processor to write a letter or a report, 
we are interested in files containing text and some 
formatting instructions, and with commands to change the 
file or to print it; fortunately, we don't have to care 
about the underlying 0's and 1's; the word-processing 
program offers us a much higher view (files, 
commands, etc.) that hides the gory underlying details.

Similarly, when we write a program, 
we usually use and/or create several layers of abstraction, 
so that, for example, once we have created a subtask that 
queries a database and stores the relevant data in 
memory, we no longer have to worry about the technical 
details of the subtask. We can use it as a sort of black 
box that will perform the desired operation for us. 
The essence of programming is to a very large extent this 
art of creating these successive layers of abstraction 
so that performing the higher level tasks becomes 
relatively easy.
\index{subtask}
\index{black box}


\section{Running Raku}
\label{running_raku}

One of the challenges of getting started with Raku is that you
might have to install Raku and related software on your computer. 
If you are familiar with your operating system, and especially
if you are comfortable with the shell or command-line interface, 
you will have no trouble installing Raku.  But for beginners, 
it can be painful to learn about system administration and 
programming at the same time.

To avoid that problem, you can start out running Raku 
in a web browser. You might want to use a search engine 
to find such a site. Currently, the easiest is probably 
to connect to the \url{https://repl.it/languages/raku} site, 
where you can type some Raku code in the main window, run 
it, and see the result in the output window below. Another 
possibility you might want to try is the \url{https://glot.io/new/raku} 
site. There are probably some others.
\index{Raku in a browser}

Sooner or later, however, you will really need to install 
Raku on your computer.

The easiest way to install Raku on your system is to 
download Rakudo Star (a distribution of Raku that contains 
the Rakudo Raku {\bf compiler}, documentation and useful modules): 
follow the instructions for your operating system at 
\url{http://rakudo.org/how-to-get-rakudo/} and at 
\url{https://raku.org/downloads/}. You may also want to look at 
third-party packages using for example Docker or Chocolately.

\index{Perl 6 version}
As of this writing, the most recent specification of 
the language is Raku version 6d (v6.d), and the most 
recent release available for download is Rakudo Star 2019.03; 
the examples in this book should 
all run with this version. You can find out the installed 
version by issuing the following command at the operating 
system prompt:
\begin{verbatim}
$ raku -v
This is Rakudo Star version 2019.03.1 built on MoarVM version 2019.03
implementing Perl 6.d.
\end{verbatim}

However, you should probably download and install the most recent 
version you can find. The output (warnings, error messages, 
etc.) you'll get from your version of Raku might in some 
cases slightly differ from what is printed in this book, 
but these possible differences should essentially be 
only cosmetic. 

Raku is derived from Perl, and it intends to carry forward the 
high ideals of the Perl community. But, compared to Perl~5, 
Raku is not just a new version of Perl. It does not 
aim at replacing Perl~5. Raku is really a new programming language, 
with a syntax that is similar to Perl~5.

The Raku {\bf interpreter} is a program that reads and 
executes Raku code. It is sometimes called REPL (for ``read, 
evaluate, print, loop''). Depending on your environment, you 
might start the interpreter by clicking on an icon, or by 
typing {\tt raku} on a command line.

When it starts, you should see output like this:
\index{interpreter}
\index{REPL}

\begin{verbatim}
To exit type 'exit' or '^D'
(Possibly some information about Raku and related software)
> 
\end{verbatim}
%

The last line with {\tt >} is a {\bf prompt} that indicates 
that the REPL is ready for you to enter code. If you type a 
line of code and hit Enter, the interpreter displays the
result: 
\index{prompt}

\begin{verbatim}
> 1 + 1
2
>
\end{verbatim}
%
You can type exit at the REPL prompt to exit the REPL.

Now you're ready to get started.
From here on, we assume that you know how to start the Raku
REPL and run code.


\section{The First Program}
\label{hello}
\index{Hello, World}

Traditionally, the first program you write in a new language
is called ``Hello, World'' because all it does is display the
words ``Hello, World.''  In Raku, it looks like this:

\begin{verbatim}
> say "Hello, World";
Hello, World
>
\end{verbatim}
%
\index{say function or method}
\index{function!say}
This is an example of what is usually called a {\bf print statement}, although it
doesn't actually print anything on paper and doesn't even 
use the {\tt print} keyword \footnote{Raku also has a {\tt print} 
function, but the {\tt say} built-in function is used here 
because it adds a new line character to the output.} (keywords are 
words which have a special meaning to the language and are 
used by the interpreter to recognize the structure of the program).  
The print statement displays a result on the screen.  In this case, 
the result is the words {\tt Hello, World}.
%
The quotation marks in the program indicate the beginning and end
of the text to be displayed; they don't appear in the result.
\index{quotation mark}
\index{print statement}
\index{statement!print}

The semi-colon ``{\tt ;}'' at the end of the line indicates 
that this is the end of the current statement. Although a 
semi-colon is technically not needed when running 
simple code directly under the REPL, it is usually 
necessary when writing a program with several lines of code, 
so you might as well just get into the habit of ending code 
instructions with a semi-colon.
\index{semi-colon}

Many other programming languages would require parentheses 
around the sentence to be displayed, but this is usually 
not necessary in Raku.

\section{Arithmetic Operators}
\index{operator!arithmetic}
\index{arithmetic operator}

After ``Hello, World,'' the next step is arithmetic.  Raku provides
{\bf operators}, which are special symbols that represent computations
like addition and multiplication.  

The operators {\tt +}, {\tt -}, {\tt *}, and {\tt /} perform addition,
subtraction, multiplication and division, as in the following examples
under the REPL:

\begin{verbatim}
> 40 + 2
42
> 43 - 1
42
> 6 * 7
42
> 84 / 2
42
\end{verbatim}
%

Since we use the REPL, we don't need an explicit print 
statement in these examples, as the REPL automatically 
prints out the result of the statements for us. In a real 
program, you would need a print statement to display 
the result, as we'll see later. Similarly, if you run 
Raku statements in the web browser mentioned in 
Section~\ref{running_raku}, you will need a 
print statement to display the result of these operations. 
For example:

\begin{verbatim}
say 40 + 2;   # -> 42
\end{verbatim}


Finally, the operator {\tt **} performs exponentiation; that is,
it raises a number to a power:

\begin{verbatim}
> 6**2 + 6
42
\end{verbatim}
%
In some other languages, the caret (``\verb"^"'') or 
circumflex accent is used for exponentiation, but in 
Raku it is used for some other purposes.
%
\index{set}
\index{set!operator}
\index{operator!set}


\section{Values and Types}
\label{values_and_types}
\index{value}
\index{type}
\index{string}

A {\bf value} is one of the basic things a program works with, like a
letter or a number.  Some values we have seen so far are {\tt 2},
{\tt 42}, and \verb'"Hello, World"'.

These values belong to different {\bf types}:
{\tt 2} is an {\bf integer}, {\tt 40 + 2} is also an integer, 
{\tt 84/2} is a {\bf rational number},
and \verb"'Hello, World'" is a {\bf string}, so called 
because the characters it contains are strung together.
\index{integer}
\index{floating-point}

If you are not sure what type a value has, Raku can
tell you:

\begin{verbatim}
> say 42.WHAT;
(Int)
> say (40 + 2).WHAT;
(Int)
> say (84 / 2).WHAT;
(Rat)
> say (42.0).WHAT
(Rat)
> say ("Hello, World").WHAT;
(Str)
>
\end{verbatim}
%
In these instructions, {\tt .WHAT} is known as an 
introspection method, that is a kind of method which 
will tell you \emph{what} (of which type) the preceding 
expression is. {\tt 42.WHAT} is an example of the dot 
syntax used for method invocation: it calls the {\tt .WHAT} 
built-in on the ``42'' expression (the invocant) and provides 
to the {\tt say} function the result of this invocation, 
which in this case is the type of the expression.
\index{WHAT}
\index{introspection}
\index{string!type}
\index{type!Str}
\index{Int type}
\index{type!Int}
\index{rational!type}
\index{type!Rat}
\index{invocant}
\index{invocation}

Not surprisingly, integers belong to the type {\tt Int},
strings belong to {\tt Str} and rational 
numbers belong to {\tt Rat}.  

Although {\tt 40 + 2} and {\tt 84 / 2} seem to yield the 
same result (42), the first expression returns an integer 
({\tt Int}) and the second a rational number ({\tt Rat}). 
The number 42.0 is also a rational.

The rational type is somewhat uncommon in most programming 
languages. Internally, these numbers are stored as two 
integers representing the numerator and the denominator 
(in their simplest terms). For example, the number 17.3 
might be stored as two integers, 173 and 10, meaning that 
Raku is really storing something meaning the $\frac{173}{10}$ 
fraction. Although this is usually not needed (except 
for introspection or debugging), you might access these 
two integers with the following methods:

\begin{verbatim}
> my $num = 17.3;
17.3
> say $num.WHAT;
(Rat)
> say $num.numerator, " ", $num.denominator; # say can print a list
173 10
> say $num.nude;      # "nude" stands for numerator-denominator
(173 10) 
\end{verbatim}
\index{numerator method}
\index{method!numerator}
\index{denominator method}
\index{method!denominator}
\index{nude method}
\index{method!nude}
%
This may seem anecdotal, but, for reasons which are 
beyond the scope of this book, this makes it possible for Raku 
to perform arithmetical operations on rational numbers with 
a much higher accuracy than most common programming languages. 
For example, if you try to perform the arithmetical operation
\verb'0.3 - 0.2 - 0.1', with most general purpose programming languages 
(and depending on your machine architecture), you 
might obtain a result such as -2.77555756156289e-17 (in Perl~5), 
-2.775558e-17 (in C under gcc), or -2.7755575615628914e-17 
(Java, Python~3, Ruby, TCL). Don't worry about these values if you 
don't understand them, let us just say that they  are 
extremely small, but they are not 0, whereas,  
obviously, the result should really be zero. In Raku, 
the result is 0 (even to the fiftieth decimal digit):
\begin{verbatim}
> my $result-should-be-zero = 0.3 - 0.2 - 0.1;
0
> printf "%.50f", $result-should-be-zero; # prints 50 decimal digits
0.00000000000000000000000000000000000000000000000000
\end{verbatim}
%
In Raku, you might even compare the result of the operation with 0:
\begin{verbatim}
> say $result-should-be-zero == 0;
True
\end{verbatim}
%
Don't do such a comparison with most common programming 
languages; you're very likely to get a wrong result.

What about values like \verb'"2"' and \verb'"42.0"'?
They look like numbers, but they are in quotation marks like
strings.
\index{quotation mark}

\begin{verbatim}
> say '2'.perl; # perl returns a representation of the invocant that could be compiled
"2"
> say "2".WHAT;
(Str)
> say '42'.WHAT;
(Str)
\end{verbatim}
%
\index{invocant}

They're strings because they are defined within quotes. Although 
Raku will often perform the necessary conversions for you, it 
is generally a good practice not to use quotation marks if your value 
is intended to be a number.

When you type a large integer, you might be tempted to use commas
between groups of digits, as in {\tt 1,234,567}.  This is not a
legal {\em integer} in Raku, but it is a legal expression:

\begin{verbatim}
> 1,234,567
(1 234 567)
>
\end{verbatim}
%
That's actually a list of three different integer numbers, and 
not what we expected at all! 

\begin{verbatim}
> say (1,234,567).WHAT
(List)
\end{verbatim}

Raku interprets {\tt 1,234,567} as a comma-separated 
sequence of three integers.  As we will see later, 
the comma is a separator used for constructing lists.
\index{comma}

You can, however, separate groups of digits with the underscore character ``\verb"_"'' for better legibility and obtain a 
proper integer:
\index{underscore character}

\begin{verbatim}
> 1_234_567
1234567
> say 1_234_567.WHAT
(Int)
>
\end{verbatim}
%

\index{sequence}



\section{Formal and Natural Languages}
\index{formal language}
\index{natural language}
\index{language!formal}
\index{language!natural}

{\bf Natural languages} are the languages people speak,
such as English, Spanish, and French.  They were not designed
by people (although people try to impose some order on them);
they evolved naturally.

{\bf Formal languages} are languages that are designed by people for
specific applications.  For example, the notation that mathematicians
use is a formal language that is particularly good at denoting
relationships among numbers and symbols.  Chemists use a formal
language to represent the chemical structure of molecules.  And
most importantly:

\begin{quote}
{\bf Programming languages are formal languages that have been
designed to express computations.}
\end{quote}

Formal languages tend to have strict {\bf syntax} rules that
govern the structure of statements.
For example, in mathematics the statement
$3 + 3 = 6$ has correct syntax, but
not $3 + = 3 \$ 6$.  In chemistry
$H_2O$ is a syntactically correct formula, but $_2Zz$ is not.
\index{syntax}

Syntax rules come in two flavors, pertaining to {\bf tokens} and
{\bf structure}.  Tokens are the basic elements of the language, such as
words, numbers, and chemical elements.  One of the problems with
$3 += 3 \$ 6$ is that \( \$ \) is not a legal token in mathematics
(at least as far as I know).  Similarly, $_2Zz$ is not legal because
there is no chemical element with the abbreviation $Zz$.
\index{token}
\index{structure}

The second type of syntax rule, structure, pertains to the way tokens are
combined.  The equation $3 += 3$ is illegal in mathematics 
because even though $+$ and $=$ are legal tokens, you can't 
have one right after the other. Similarly, in a chemical formula, 
the subscript representing the number of atoms in a 
chemical compound comes after the element name, not before.

This is @ well-structured Engli\$h
sentence with invalid t*kens in it.  This sentence all valid 
tokens has, but invalid structure with.

When you read a sentence in English or a statement in a formal
language, you have to figure out the structure
(although in a natural language you do this subconsciously).  This
process is called {\bf parsing}.
\index{parse}

Although formal and natural languages have many features in
common---tokens, structure, and syntax---there are some
differences:
\index{ambiguity}
\index{redundancy}
\index{literalness}

\begin{description}

\item[Ambiguity] Natural languages are full of ambiguity, which
people deal with by using contextual clues and other information.
Formal languages are designed to be nearly or completely unambiguous,
which means that any statement has exactly one meaning. 

\item[Redundancy] In order to make up for ambiguity and reduce
misunderstandings, natural languages employ lots of
redundancy.  As a result, they are often verbose.  Formal languages
are less redundant and more concise.

\item[Literalness] Natural languages are full of idiom and metaphor.
If we say, ``The penny dropped,'' there is probably no penny and
nothing dropping (this idiom means that someone understood something
after a period of confusion).  Formal languages
mean exactly what they say.

\end{description}

Because we all grow up speaking natural languages, it is sometimes
hard to adjust to formal languages.  The difference between formal and
natural language is like the difference between poetry and prose, but
more so: \index{poetry} \index{prose}

\begin{description}

\item[Poetry] Words are used for their sounds as well as for
their meaning, and the whole poem together creates an effect or
emotional response.  Ambiguity is not only common but often
deliberate.

\item[Prose] The literal meaning of words is more important,
and the structure contributes more meaning.  Prose is more amenable to
analysis than poetry but still often ambiguous.

\item[Programs] The meaning of a computer program is unambiguous
and literal, and can be understood entirely by analysis of the
tokens and structure.

\end{description}

Formal languages are more dense
than natural languages, so it takes longer to read them.  Also, the
structure is important, so it is not always best to read
from top to bottom, left to right.  Instead, learn to parse the
program in your head, identifying the tokens and interpreting the
structure.  Finally, the details matter.  Small errors in
spelling and punctuation, which you can get away
with in natural languages, can make a big difference in a formal
language.


\section{Debugging}
\index{debugging}

Programmers make mistakes.  Programming errors are usually 
called {\bf bugs} and the process of tracking them down is called
{\bf debugging}.
\index{debugging}
\index{bug}

Programming, and especially debugging, sometimes brings out strong
emotions.  If you are struggling with a difficult bug, you might 
feel angry, despondent, or embarrassed.

There is evidence that people naturally respond to computers as if
they were people.  When they work well, we think
of them as teammates, and when they are obstinate or rude, we
respond to them the same way we respond to rude,
obstinate people\footnote{Reeves and Nass, {\it The Media
    Equation: How People Treat Computers, Television, and New Media
    Like Real People and Places}, (Center for the Study of Language and Information, 2003).)}.
\index{debugging!emotional response}
\index{emotional debugging}

Preparing for these reactions might help you deal with them.
One approach is to think of the computer as an employee with
certain strengths, like speed and precision, and
particular weaknesses, like lack of empathy and inability
to grasp the big picture.

Your job is to be a good manager: find ways to take advantage
of the strengths and mitigate the weaknesses.  And find ways
to use your emotions to engage with the problem,
without letting your reactions interfere with your ability
to work effectively.

Learning to debug can be frustrating, but it is a valuable skill
that is useful for many activities beyond programming.  At the
end of each chapter there is a section, like this one,
with our suggestions for debugging.  I hope they help!


\section{Glossary}

\begin{description}

\item[Problem solving]  The process of formulating a problem, finding
a solution, and expressing it.
\index{problem solving}

\item[Abstraction] A way of providing a high-level view 
of a task and hiding the underlying technical details so 
that this task becomes easy.

\item[Interpreter]  A program that reads another program and executes it
\index{interpret}

\item[Compiler]  A program that reads another program and 
transforms it into executable computer code; there used to be 
a strong difference between interpreted and compiled languages, 
but this distinction has become blurred over the last 
two decades or so.
\index{compiler}

\item[Prompt] Characters displayed by the interpreter to indicate
that it is ready to take input from the user.
\index{prompt}

\item[Program] A set of instructions that specifies a computation.
\index{program}

\item[Print statement]  An instruction that causes the Raku
interpreter to display a value on the screen.
\index{print statement}
\index{statement!print}

\item[Operator]  A special symbol that represents a simple computation like
addition, multiplication, or string concatenation.
\index{operator}

\item[Value]  One of the basic units of data, like a number or string, 
that a program manipulates.
\index{value}

\item[Type] A category of values.  The types we have seen so far
are integers (type {\tt Int}), rational numbers (type {\tt
Rat}), and strings (type {\tt Str}).
\index{type}

\item[Integer] A type that represents whole numbers.
\index{integer}

\item[Rational] A type that represents numbers with fractional
parts. Internally, Raku stores a rational as two integers 
representing respectively the numerator and the denominator 
of the fractional number.
\index{rational}

\item[String] A type that represents sequences of characters.
\index{string}

\item[Natural language]  Any one of the languages that people speak that
evolved naturally.
\index{natural language}

\item[Formal language]  Any one of the languages that people have designed
for specific purposes, such as representing mathematical ideas or
computer programs; all programming languages are formal languages.
\index{formal language}

\item[Token]  One of the basic elements of the syntactic structure of
a program, analogous to a word in a natural language.
\index{token}

\item[Syntax] The rules that govern the structure of a program.
\index{syntax}

\item[Parse] To examine a program and analyze the syntactic structure.
\index{parse}

\item[Bug] An error in a program.
\index{bug}

\item[Debugging] The process of finding and correcting bugs.
\index{debugging}

\end{description}


\section{Exercises}

\begin{exercise}

It is a good idea to read this book in front of a computer so you 
can try out the examples as you go.

Whenever you are experimenting with a new feature, you should try
to make mistakes.  For example, in the ``Hello, world!'' program,
what happens if you leave out one of the quotation marks?  What
if you leave out both?  What if you spell {\tt say} wrong?
\index{error message}

This kind of experiment helps you remember what you read; it also
helps when you are programming, because you get to know what the error
messages mean.  It is better to make mistakes now and on purpose than
later and accidentally.

Please note that most exercises in this book are provided with 
a solution in the appendix. However, the exercises in this chapter   
and in the next chapter are not intended to let you solve an 
actual problem but are designed to simply let you experiment 
with the Raku interpreter; there is no good solution, just try 
out what is proposed to get a feeling on how it works.

\begin{enumerate}

\item If you are trying to print a string, what happens if you
leave out one of the quotation marks, or both?

\item You can use a minus sign to make a negative number like
{\tt -2}.  What happens if you put a plus sign before a number?
What about {\tt 2++2}?

\item In math notation, leading zeros are OK, as in {\tt 02}.
What happens if you try this in Raku?

\item What happens if you have two values with no operator
between them, such as {\tt say 2 2;}?

\end{enumerate}

\end{exercise}



\begin{exercise}

Start the Raku REPL interpreter and use it as a calculator.

\begin{enumerate}

\item How many seconds are there in 42 minutes, 42 seconds?

\item How many miles are there in 10 kilometers?  Hint: there are 1.61
  kilometers in a mile.

\item If you run a 10 kilometer race in 42 minutes, 42 seconds, what is
  your average pace (time per mile in minutes and seconds)?  What is
  your average speed in miles per hour?

\index{calculator}
\index{running pace}

\end{enumerate}

\end{exercise}


