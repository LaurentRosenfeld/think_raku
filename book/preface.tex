
\chapter{Preface}

Welcome to the art of computer programming and to the 
new Raku language. This is one of the first books using Raku,
a powerful, expressive, malleable and highly extensible 
programming language. But this book is less 
about Raku, and more about learning 
how to write programs for computers. 

This book is intended for beginners and does not require 
any prior programming knowledge, but it is my hope 
that even those of you with programming experience will 
benefit from reading it.

\section*{The Aim of this Book}

This aim of this book is not primarily to teach Raku, 
but instead to teach the art 
of programming, using the Raku language. After having 
completed this book, you should hopefully be able 
to write programs to solve relatively difficult problems in 
Raku, but my main aim is to teach computer science, software 
programming, and problem solving rather than solely to teach 
the Raku language itself. 

This means that I will not cover every aspect of Raku, but 
only a (relatively large, but yet incomplete) subset of it. 
By no means is this book intended to be a reference on the 
language.

It is not possible to learn programming or to learn a new 
programming language by just reading a book; practicing 
is essential. This book contains a lot of exercises. You 
are strongly encouraged to make a real effort to do them. And, 
whether successful or not in solving the exercises, you 
should take a look at the solutions in the Appendix, 
as, very often, several solutions are suggested with further 
discussion on the subject and the issues involved. Sometimes, the solution 
section of the Appendix also introduces examples of topics 
that will be covered in the next chapter--and sometimes even 
things that are not covered elsewhere in the book. So, to get 
the most out the book, I suggest you try to solve the exercises 
as well as review the solutions and attempt them.

There are more than one thousand code examples in this book; 
study them, make sure to understand them, and run them. When 
possible, try to change them and see what happens. You're 
likely to learn a lot from this process.


\section*{The History of this Book}

There is a simple thing that you need to know in order to 
understand the coming paragraphs. The Raku programming language 
has not always been called Raku. The project started under the 
Perl~6 name, because it was originally thought to be the next version 
of Perl, a successor to Perl~5. With time, however, the language 
has grown to be syntactically quite different from Perl, although 
keeping the spirit of Perl and carrying forward the ideals of 
the Perl Community. Because of these differences, it was decided 
in October 2019 to rename the Perl~6 language into Raku. The book 
that you are reading was originally called \emph{Think Perl~6}. 
This new edition thus reflects the re-branding of the language.

In the course of the three to four years before the beginning of 2016, I  
had translated or adapted to French a number of tutorials 
and articles on what was still called at the time Perl~6, and 
I'd also written a few entirely 
new ones in French.~\footnote{See for example 
\url{http://perl.developpez.com/cours/\#TutorielsPerl6}.} 
Together, these documents represented by the end of 2015 
somewhere between 250 and 300 pages of material on Perl~6. 
By that time, I had 
probably made public more material on this new language in French than 
all other authors taken together.

In late 2015, I began to feel that a Perl~6 document for beginners 
was something missing that I was willing to undertake. 
I looked around and found that it did not seem to 
exist in English either. I came to the idea that, after all, 
it might be more useful to write such a document initially 
in English, to give it a broader audience. I started contemplating 
writing a beginner introduction to Perl~6 
programming. My idea at the time was something like a 50- to 
70-page tutorial and I started to gather material and ideas 
in this direction.

Then, something happened that changed my plans.

In December 2015, friends of mine were contemplating translating 
into French Allen B. Downey's \emph{Think Python, Second Edition}\footnote{See \url{http://greenteapress.com/wp/think-python-2e/}.}. 
I had read an earlier edition of that excellent book and fully supported 
the idea of translating it\footnote{I know, it's 
about Python, not Perl or Raku. But I don't believe in engaging 
in ``language wars'' and think that we all have to learn from 
other languages; to me, Perl's motto, ``there is more than 
one way to do it,'' also means that doing it in Python (or some 
other language) is truly an acceptable possibility.}. As it 
turned out, I ended up being a co-translator and the technical 
editor of the French translation of that book\footnote{See 
\url{http://allen-downey.developpez.com/livres/python/pensez-python/}.}.

While working on the French translation of Allen's Python book, 
the idea came to me that, rather than writing a tutorial on 
Perl~6, it might be more useful to make a ``Perl~6 translation'' 
of \emph{Think Python}. Since I was in contact with Allen in the context 
of the French translation, I suggested this to Allen, who 
warmly welcomed the idea. This is how I started to write this 
book late January 2016, just after having completed the 
work on the French translation of his Python book.

I had to use the ``Perl~6'' name in the preceding paragraphs, since 
it is the way it was named at that point in history. From now on, 
I'll use the new name of the language, \emph{Raku}, unless referring 
to events prior to the renaming.

This book is thus largely derived on Allen's \emph{Think Python}, 
but adapted to Raku. As it happened, it is also much more 
than just a ``Raku translation'' of Allen's book: with 
quite a lot of new material, it has become a brand new book, 
largely indebted to Allen's book, but yet a new book for which 
I take all responsibility. Any errors are mine, 
not Allen's.

My hope is that this will be useful to the Raku community, and 
more broadly to the open source and general 
computer programming communities. In an interview with 
\emph{LinuxVoice} (July 2015), Larry Wall, the creator of Raku, 
said: ``We do think that Perl~6 will be learnable as a first language.''
Hopefully this book will contribute to making this happen. 

\section*{Acknowledgments}

I just don't know how I could thank Larry Wall to the level of 
gratitude that he deserves for having created Perl in the first 
place, and Raku more recently. Be blessed for eternity, Larry, 
for all of that. 

And thank to you all of you who took part to this 
adventure (in no particular order), Tom, Damian, 
chromatic, Nathan, brian, Jan, Jarkko, John, Johan, Randall, 
Mark Jason, Ovid, Nick, Tim, Andy, Chip, Matt, Michael, Tatsuhiko, 
Dave, Rafael, Chris, Stevan, Saraty, Malcolm, Graham, Leon, 
Ricardo, Gurusamy, Scott and too many others to name.  

All my thanks also to those who believed in 
this Perl~6 and then Raku project and made it happen, including those who 
quit at one point or another but contributed for some 
time; I know that this wasn't always easy.

Many thanks to Allen Downey, who very kindly supported my idea of 
adapting his book to this language and helped me in many respects, but 
also refrained from interfering in what I was putting into 
this new book.

I very warmly thank the people at O'Reilly who accepted the 
idea of this book and suggested many corrections or 
improvements. I want to thank especially 
Dawn Schanafelt, my editor at O'Reilly, whose advice 
has truly contributed to making this a better book. Many 
thanks also to Charles Roumeliotis, the copy editor, and 
Kristen Brown, the production editor, who fixed many 
typographical problems and spelling mistakes.

Thanks a lot to readers who have offered comments 
or submitted suggestions or corrections, as well as encouragements.

If you see anything that needs to be corrected or that 
could be improved, please kindly send your comments to 
\url{think.perl6 (at) gmail.com}.
% ...


\section*{Contributor List}

% ...
I would like to thank especially Moritz Lenz and Elizabeth 
Mattijsen, who reviewed in detail drafts of this book 
and suggested quite a number of improvements and corrections. Liz 
spent a lot of time on a detailed review of the full 
content of this book and I am especially grateful to her for 
her numerous and very useful comments. Thanks also to Timo Paulssen and 
ryanschoppe who also reviewed early drafts and provided some  
useful suggestions. Many thanks also to Uri Guttman, who reviewed 
this book and suggested a number of small corrections and improvements 
shortly before publication. 

Kamimura, James Lenz, and Jeff McClelland each submitted a couple 
of corrections in the errata list on the O'Reilly web site.
zengargoyle pointed out a spurious character in a regex 
and fixed it in the GitHub repository of the book. zengargoyle
also suggested a clarification in the chapter about functional 
programming. Another James (second name unknown to me) 
submitted an erratum on the O'Reilly web site. Mikhail Korenev 
suggested accurate corrections to three code samples.  
Sébastien Dorey, Jean Forget, and Roland Schmitz sent some e-mails 
suggesting a few useful corrections or improvements.
Luis F. Uceta translated this book into Spanish and found in the 
process quite a few typos he corrected on the Github repository. 
Gil Magno, zoffixznet and Joaquin Ferrero also suggested some 
corrections on Github. Threadless-screw fixed a couple of 
faulty cross-references in the chapter about hashes. 
Boyd Duffee spotted two code samples that used to work fine when 
I wrote them four years ago, but no longer work, due to some 
implementation changes. leszekdubiel suggested an improvement 
that probably clarifies a bit the meaning of a sentence.


\clearemptydoublepage

% TABLE OF CONTENTS
\begin{latexonly}

\tableofcontents

\clearemptydoublepage

\end{latexonly}

